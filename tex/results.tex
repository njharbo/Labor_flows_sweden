\section{Results}
\label{sec:Results}

\begin{table}[h]
\caption{\label{tab:decomp} Decomposition of the variance in unemployment, 1987-2012}
\begin{tabularx} {1\textwidth} { l l X cXcXcXcX|c} \\ \hline
\multicolumn{11}{c}{\textbf{A. Non steady state method}} \\ \hline

 \multicolumn{2}{l}{ \backslashbox{From}{To}}            & & PE   & & TE    & & U     & & I     & & $\sum$\\\hline

\input{../../programs/decompose/matlab/tables/decomp_nonss.tex} 
\hline

\multicolumn{12}{c}{\textbf{B. Steady state method}}\\\hline

 \multicolumn{2}{l}{ \backslashbox{From}{To}}     & & PE   & & TE    & & U     & & I     & & $\sum$\\\hline

\input{../../programs/decompose/matlab/tables/decomp_ss.tex}
\hline

\multicolumn{12}{c}{\textbf{C. Difference (panel B. - panel A.) }}\\\hline

\multicolumn{2}{l}{ \backslashbox{From}{To}}    &  & PE   & & TE    & & U     & & I     & & $\sum$\\\hline

\input{../../programs/decompose/matlab/tables/decomp_dif.tex}
\hline\hline

\multicolumn{12}{l}{\footnotesize \emph{Notes:} Panel A and B are computed using \eqref{eq:beta_nonstst} and \eqref{eq:beta_stst}, respectively. PE: Permanent employment. TE: Temporary }\\

\multicolumn{12}{l}{\footnotesize employment. U: Unemployment. I: Inactivity. The total variation has been normalized to the contribution from}\\

\multicolumn{12}{l}{\footnotesize  flows from/to these 4 states.}\\

\multicolumn{12}{l}{\footnotesize \emph{Source:} Own calculations on data from Statistics Sweden}\\
\end{tabularx}

\end{table}

Panel A in Table \ref{tab:decomp} shows the decomposition of the variation in the unemployment share [TBD: Fix to pct of labor force] using the non steady state method described in Section \ref{sec:method_non_st_st}. This is our preferred method. A number of observations can be made from this panel. The contribution from the variation in the \emph{inflow} to unemployment (51 \%) is marginally larger than the variation in the \emph{outflow} from unemployment (45 \%). The largest contribution from the variation in \emph{inflow} to unemployment stems from temporary employment (19 \%), but the contributions from inactivity and permanent employment is only marginally smaller (17 \% and 15 \%, respectively). The lion share of the variation in \emph{outflow} from unemployment also comes from temporary employment (18 \%), while the contributions from outflow from inactivity and permanent employment are somewhat smaller (12 \% and 14 \%, respectively). Finally, note that the contributions from flows that do not involve unemployment are only marginal.

We also compute the decomposition using the traditional steady state method. As explained above in Section \ref{sec:method} this method is only suitable insofar as if convergence to the steady state can be assumed to be fast. As noted in Section \ref{sec:method_st_st_conv_speed} this convergence rate can be calculated as the second largest eigenvalue to the transition matrix, $Q(t)$. We plot these convergence rates in Figure TBD. As they are around $0.03$ the halving time to a deviation from the steady state is approx. $70/3 \approx$ 23 months $\approx 2$ years. Hence, the steady state method is unlikely to be useful for our data. Nevertheless, we will apply it to gauge the size of the discrepancy to the non-steady state method.

In Panel B of Table \ref{tab:decomp} the variation decomposition is done using the steady state method. The method yields exactly the same contribution from variation in the \emph{inflow} from unemployment, although it overestimates the contribution from permanent employment and underestimates the contribution from temporary employment. The contribution from the variation in outflow from unemployment is underestimated, due to an underestimation of the contribution from the variation in outflow to inactivity. Finally, the contribution coming from the variation in inflow to permanent employment is substantially overestimated due to an overestimation of the contribution from temporary employment.

Why is are the contribution stemming from permanent employment overestimated? The reason is the at the stock of persons in permanent emeployment is sizable (55-65\% of the population). Consequently, small swings in the in- and outflow rates from this state will have large impact on the steady state level of unemployment. However, as the flows in and out of permanent employment are low it takes a long time to reach these new steady states. Consequently, short-lived swings in the transition in and out of regular employment will cause large variation in the steady state but little in the actual distribution. 

Taken together our results suggests that properly accounting for out of steady state dynamics is important when analyzing a dual labor market. In a Swedish context failing to do so leads to a substantially underestimation of the variation coming from temporary employment. We think this point is also important in a broader European context, where labor markets have a substantial share of temporary contracts [TBD: Cite OECD employment outlook]. When analyzing labor market duality in France and Spain, existing studies \cite{Hairault2015, Silva2013} have not accounted for the out of steady state dynamics. We think this could explain why they find rather modest contributions from temporary contracts [TBD: Numbers].