\begin{abstract}
Which labor market flows drive the business cycle variability of unemployment in Sweden? To answer this question we develop a new decomposition method allowing for slow convergence to steady state and an arbitrary number of labor market states. Applying this method to new data from the Swedish Labor Market Survey covering the period 1987-2012 we show that the contributions from inflow to and outflow from unemployment is roughly 50/50. Decomposing further we show that approx. 40 \% of the variation stems from flows in and out of temporary contracts. We further show that not accounting for out-of-steady-state dynamics substantially biases the contributions from temporary contracts downwards. This finding calls into doubt results from existing studies on dual labor markets in Europe. 
\end{abstract}
