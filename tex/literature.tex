\section{Literature}

The debate on the relative role of job-finding and seperation rates in explaining fluctuations in unemployment is long, but has been revived by an influential paper by \cite{Shimer}. He investigates US data and finds substantial procyclical variations in the job-finding rates, while the job-seperation rates is relatively acyclic. 

These findings have been put into question by \cite{Elsby et al}. These authors use the same data as \cite{Shimer}, but uses a log decomposition of the unemployment level. Using this decomposition they find counter-cyclical seperation rates as well as countercyclical job-finding rates. 