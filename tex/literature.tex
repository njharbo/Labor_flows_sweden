\section{Literature}

Our study relates to a body of literature, which discuss how much of the variation in unemployment is caused by inflow into unemployment and outflow from unemployment, respectively. 

This literature goes back to \cite{Darby1986} who construct a framework for decomposing the variation in the level of unemployment in the United States. Specifically, they show how the current level of unemployment can be written as a function of an initial level and subsequent in- and outflow rates. They apply this framework to the monthly flow rates between labor market states, constructed from the Current Population Survey. Using this setup they construct two counterfactual time-series for unemployment: One where the inflow rate into unemployment is held fixed, and one where the outflow rate from unemployment is held fixed. The ratio of variance in these series to the variance of actual unemployment is then computed, and it is shown that this ratio is higher for the counterfactual serie where the outflow rate is held constant. \cite{Darby1986} take this as evidence for fluctuations in the inflow rate being the driving factor behind the observed variation in the level of unemployment. The result is strengthened when the authors account for compositional effects.\footnote{The study does this by regressing the in- and outflow rates on compositional factors as well as lagged flow  rates. Using the resulting regressors counterfactual flow rates are constructed, where the rates only vary due to composition. Using these counterfactual flow rates the two counterfactual time-series for unemployment are again created: one where the inflow rate is held constant, and one where the outflow rate is held constant.}  The authors take the overall finding to be important, as the contemporaneous macroeconomic literature focused on variations in the inflow to unemployment in explaining fluctuations in unemployment[TBD: Cite Dornbusch/fisher or Gordon]. Here a recession was characterised by a downwards shift in the wage-offer distribution. Being unaware of this shift, would then be more likely to decline incoming job-offers why outflows from unemployment would diminish. The findings by \cite{Darby1986} go against this story.

A supporting result is found in \cite{Blanchard1990}. They also construct gross worker flows from CPS data, as well as gross flows in and out of manufactoring employment from firm data. Using this data the paper focuses on two aspect of the labor market. First, it focuses on the creation and destruction of jobs over the business cycle. Here the paper finds that reduced employment in recessions(booms) are more driven by higher(lower) job-destruction rates than of lower(higher) job-creation rates. Second, the paper maps the flows of workers between different labor market states. Specifically, the flow
from employment to unemployment is found to increase in a recession, while the flow from employment to out-side the labor market decreases. Conversely, the flow from unemployment to employment is found to increase in a recession, while the flow from outside the labor force to employment decreases. 

The take-away from these early paper was thus, that the \emph{The ins win} in the sense that variations in the inflow to unemployment is the predominant reason behind the observed variation in the level of unemployment.

NOTE: \cite{Darby1985} paper is not related

Later these findings
  
decompose the fluctuations in the aggregate level of unemployment.  

Early literature \cite{Darby1986} \cite{Darby1986} \cite{Davis1992} \cite{Blanchard1990}

Challenged by \cite{Shimer2012} (circulated as \cite{Shimer2007}) and \cite{Hal2005a} and \cite{Hal2005b}.

\cite{Elsby2009} \cite{Fujita2009} \cite{Yashiv2007} challenges findings by Shimer and Hall. 

Other studies have done similar exercises for other countries. \cite{Gomes2012} for United Kingdom. \cite{Silva2013} for Spain. \cite{Petrongolo2008} for France, Spain and United Kingdom. \cite{Elsby2013} for OECD countries. 

The debate on the relative role of job-finding and seperation rates in explaining fluctuations in unemployment is long, but has been revived by an influential paper by \cite{Shimer}. He investigates US data and finds substantial procyclical variations in the job-finding rates, while the job-seperation rates is relatively acyclic. 

These findings have been put into question by \cite{Elsby et al}. These authors use the same data as \cite{Shimer}, but uses a log decomposition of the unemployment level. Using this decomposition they find counter-cyclical seperation rates as well as countercyclical job-finding rates. 