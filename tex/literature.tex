\section{Literature}

Our study relates to a body of literature, which discusses the decomposition of variation in unemployment into variation coming from inflow into and outflow from unemployment, respectively. 

The findings from the early literature can be summarised in the catch-phase \emph{the ins win}.\footnote{This sentence is coined by \cite{Darby1986}} This phase summarises the notion that variations in the inflow to unemployment accounts for the lions share of the observed variation in the unemployment.

This result was found by \cite{Darby1986} who construct a framework for decomposing the variation in the level of unemployment in the United States. Specifically, they show how the current level of unemployment can be written as a function of an initial level and subsequent in- and outflow rates. They apply this framework to the monthly flow rates between labor market states, constructed from the Current Population Survey. Using this setup they construct two counterfactual time-series for unemployment: One where the inflow rate into unemployment is held fixed, and one where the outflow rate from unemployment is held fixed. The ratio of variance in these series to the variance of actual unemployment is then computed, and it is shown that this ratio is higher for the counterfactual serie where the outflow rate is held constant. \cite{Darby1986} take this as evidence for fluctuations in the inflow rate being the driving factor behind the observed variation in the level of unemployment. The result is strengthened when the authors account for compositional effects.\footnote{The study does this by regressing the in- and outflow rates on compositional factors as well as lagged flow  rates. Using the resulting regressors counterfactual flow rates are constructed, where the rates only vary due to composition. Using these counterfactual flow rates the two counterfactual time-series for unemployment are again created: one where the inflow rate is held constant, and one where the outflow rate is held constant.}  The authors take the overall finding to be important, as the contemporaneous macroeconomic literature focused on variations in the inflow to unemployment in explaining fluctuations in unemployment[TBD: Cite Dornbusch/fisher or Gordon]. Here a recession was characterised by a downwards shift in the wage-offer distribution. Being unaware of this shift, would then be more likely to decline incoming job-offers why outflows from unemployment would diminish. The findings by \cite{Darby1986} go against this story.

More supporting evidence of the \emph{the ins win} hypothesis is found by \cite{Blanchard1990}. They also construct gross worker flows from CPS data, as well as gross flows in and out of manufactoring employment from firm data. Using this data the paper focuses on two aspect of the labor market. First, it focuses on the creation and destruction of jobs over the business cycle. Here the paper finds that reduced employment in recessions(booms) are more driven by higher(lower) job-destruction rates than of lower(higher) job-creation rates. Second, the paper maps the flows of workers between different labor market states. Specifically, the flow from employment to unemployment is found to increase in a recession, while the flow from employment to out-side the labor market decreases. Conversely, the flow from unemployment to employment is found to increase in a recession, while the flow from outside the labor force to employment decreases. 

NOTE: \cite{Darby1985} paper is not related

Later, \cite{Shimer2012}\footnote{The first version of this paper appeared in 2005 \citep{Shimer2005b}.}  challenged these results. This paper uses information on the unemployment duration as well as the stock of unemployed from the Current Population Survey (CPS) and develops a continous time model, where data is observed at discrete times. This allows for the correction of time-aggregation issues. Data is de-trended using HP-filtering. To decompose the fluctuations in employment into contributions stemming from seperation and job-finding rates, the paper uses that the steady state level of unemployment can be written as a function of the job-finding and seperation rate. This steady state level of unemployment can be shown to track the actual level closely. The expression also allow for the construction of two counterfactual unemployment rates: one where the seperation rate is held constant at its average level and the job-finding rate is allowed to vary, and one where the opposite is true. The variance contribution from each flow rate is then gauged by computing the ratio of the covariance between the counter-factual and the actual unemployment rate to the variance of the actual unemployment rate. Doing so the paper finds that the job-finding rate accounts for two-thirds of the variation in the unemployment rate. This result is at odds with the previous literature, and \cite{Shimer2012} argues that two reasons are behind this. First, fluctuations in the inflow to unemployment has become quantitatively less important in the last two decades. Second, ignoring the issue of time-aggregation will bias the analysis towards finding a counter-cyclical inflow rate into unemployment. This is because a lower outflow rate \emph{ceteris paribus} makes it more likely that a unemployment worker is  \emph{measured} as being unemployment during a spell of unemployment.

Supporting evidence for for this result is found by \cite{Hall2005}. He inspects movements in the seperation rate using six different data sources. First, the Job Openings and Labor Turnover Survey (JOLTS), which since 2000 measures the seperation rate from the firm side. Second, he construct a time-serie going back to 1948 the seperation rate by regressing the seperation rate in the JOLTS on  industry-growth rates, and then using the historical industry-growth rates to construct a computed seperation rate before 2000. Third, he presents Shimer's measure of seperation from the CPS. Fourth, he measures seperations directly from the CPS via a question included since 1994. Fifth, measures the seperation rate from the Survey of Income and Program Participation, which covers the period 1983 to 1995 and surveys 30 000 workers.\footnote{This data is originally compiled by \cite{Gottschalk2000}.} Finally, he uses data from the hiring rates in the CPS, which traditionally is closely correlated with the seperation rate.  Hall's conclusion from this inspection of evidence is that a constant seperation rate over the last decades is a good approximation. %, although he notes that the evidence is not strong. 

\cite{Fujita2008} on the other hand find that the seperation rate plays a larger role in explaining the unemployment variability. This paper takes a different approach to \cite{Shimer2012} both in terms of data and and methodology. In terms of data they use CPS data on the individual level. From this compute month-to-month transitions and correct for bias stemming from margin error\footnote{Due to sample rotation and temporary absence of individuals transition information is unavailable for a subset of the sample} as well as time-aggregation. In terms of methodology, they also rely on the steady state approximation on unemployment from \cite{Shimer2012}, but instead of constructing counter-factorial time-series they log-linearise the steady state equation and do a traditional variance decomposition of the resulting terms. They take out the high-frequency trend of the time-series by means of HP-filtering or first differencing. Doing so they find that 40-50 percent of the business cycle fluctuation in unemployment can be explained by the seperation rate. Although, this is somewhat higher than the share reported by \cite{Shimer2012} the two papers both find that the contribution from the seperation rate has declined in the last two decades.

\cite{Elsby2009b} also find a somewhat larger role for seperations than \cite{Shimer2012} Like \cite{Shimer2005} they rely on data from the repeated cross-section of the CPS rather than the gross-flow data used by \cite{Fujita2008}. [TBD: Describe differences in time-aggregation method compared to Shimer] They also take point of departure in the observation, that the actual level of unemployment can be well approximated using the expression for steady state unemployment which is written as a function of the seperation and job-finding rate. This expression can be log-linearised, which allows for the calculation of contributions unemployment changes coming from changes in the separation and job-finding rate, respectively. They do this decomposition for 10 recessions covering the period 1948-2004, and show that on average approximately 35 percent of the increase in unemployment came from higher inflow while the remaining 65 percent came from lower outflow. This split has however changed over time, as the contribution coming from higher seperation has been lower in more recent recessions. 

\cite{Yashiv2007}

In addition to the literature focusing on the United States, a number of studies have studied the role of seperation and job-finding rates in other other countries. 

\cite{Petrongolo2008} study France, Spain and United Kingdom. They use administrative data on the stock and inflow of unemployed workers as well as data on gross-flows from labor force surveys. Using the time-aggregation correction from \cite{Shimer2012} and the decomposition method from \cite{Fujita2009} they show that (i) approx. 30 \% of the historical variation in UK unemployment is explained by variation in the inflow to unemployment, (ii) in France only 20 \% of the the variation is driven the inflow, while (iii) the variation in Spain is explained evenly by inflow and outflow.  

\cite{Elsby2013} study 14 OECD countries. They use yearly data on the stock of unemployed workers broken down by duration. The data stems from national labor force surveys but is compiled the OECD. For time-aggregation they take point of departure in the method  from \cite{Shimer2012}, but extend it to account for noisy time-series of short-run unemployed. The idea is that surveys in countries with a low proportion of short-run unemployed might measure the stock of short-run unemployed with a high variance. To account for this they propose to estimate the job-finding rate not only by means of the stock of total and short-run unemployment, but by means of the entire distribution of unemployment. The underlying assumption for this approach to yield a consistent estimate of the average outflow rate among the unemployed is that the job-finding rate is not duration dependent. They test this for all countries, and apply the extended method where the test can be accepted and the approach from \cite{Shimer2012} where the test is rejected. In addition, they develop a method that allows for a decomposition of the variance in unemployment in cases where the actual level of unemployment is not well-approximated by the steady state. In such cases, changes in unemployment will be a function of (i) convergence towards the steady state and (ii) changes in the level of steady state. They find that unemployment variation in Anglo-Saxon economies approximately is explained by a 15:85 inflow-outflow split, while the approximate split in continental Europe is 45:55.

\cite{Gomes2012} studies the United Kingdom. He uses gross-flow time series from the labor force survey, corrects these for the time-aggregation problem by means of the method proposed by \cite{Shimer2012} and decomposes the variation in unemployment both using the method from \cite{Shimer2012} as well as from \cite{Fujita2008}. He finds that the in- and outflow are roughly equally important. 

\cite{Silva2013} analyses Spain. They use gross-flow rates from the Spanish labor force survey and correct for time-aggregation using XXXX. Then, the set up a 4 state model, allowing for both regular and temporary employment, and uses this to decompose the variation in unemployment using both the method proposed by \cite{Shimer2012} and \cite{Fujita2009}. They find that temporary jobs explain a large part of the the fluctuations in the unemployment rate. Specifically, the transition rates involving temporary contracts accounts for roughly 60 \% of the fluctuation in the unemployment rate.