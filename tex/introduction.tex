\section{Introduction}
\label{sec:intro}

Inflows to unemployment and outflows from unemployment jointly determine the unemployment rate. This motivates the literature that tries to decompose the variation in unemployment into one part due to inflows and one part due to inflows. The standard method in this literature builds on \cite{Shimer2012}. He notes that actual US unemployment is close to the steady state implied by the prevailing flow rates, and use this observation to decompose unemployment variations by analyzing how inflows and outflows impact the steady state rate of unemployment. A number of responses to \cite{Shimer2012} have taken the same approach when analyzing the US labor market \cite{Fujita2009, Elsby2009b}.\footnote{\cite{Shimer2012} was first published as a working paper in 2005 \cite{Shimer2005b}.}

However, when working with European data there are problems in applying the steady state approach. First, gross flows are much smaller in European countries which means that the steady state unemployment rate is not always a good approximation of the actual unemployment rate \citep{Elsby2013}. Second, many European labor markets are characterized by duality, which means that there potentially are large differences in labor market dynamics between permanent and temporary jobs. This 
duality makes it potentially important to distinguish between permanent and temporary jobs in the decomposition analysis. 

Recent papers have started to address these problems. Regarding slow convergence to steady state, \citep{Elsby2013} develop a two stage model and show how to decompose the variability in unemployment without imposing the steady state assumption. This is useful when convergence to steady state is slow, and the steady state assumption thus is at odds with the data. They use their method to decompose variations in unemployment across OECD countries. Regarding labor market duality \cite{Silva2013} uses a four state model (inactivity, unemployment, regular employment and temporary employment) to decompose the labor market flows in Spain. However, this model still relies on the steady state assumption. 

This paper makes a methodological contribution to the literature by developing a model that allows jointly for (i) slow convergence to steady state and (ii) an arbitrary number of labor market states. To do so we take part of departure in the assumption that the observed data is generated by a continuous time Markov chain. This formulation allows us to formulate the current level of unemployment as a function of the full history of flow-rates between labor market states. Log-linearizing this expression further allows for a decomposition of the variance in unemployment into contributions from all labor market flows. We will further contrast our results with the results from the standard steady state method.

We apply our methodology to the Swedish labor market through the period 1987-2011, which has not previously been analysed. Here convergence to the steady state is slow, with a half life of deviations from steady state of approx. 2 years, and the labor market is dual as approx. 16 \% [Check!] of employment being on temporary contracts. Doing so we find that approx. 50 \% of the variation in unemployment can be contributed to variability in the \emph{inflow} to unemployment, while the remaining 50 \% can be contributed to variability in the \emph{outflow} from unemployment. Decomposing further we find that temporary contracts accounts for approx. 40 \% of the variability, with half of this coming from the inflow to unemployment from temporary contracts and half stemming from the flow in the opposite direction. This is sizable given that temporary contracts only account for approx. 16 \% [TBD: CHECK] of all employment. We also show that properly accounting for the out of steady state dynamics is quantitatively important. Indeed, in the case of Sweden using the standard decomposition from the literature, which relies on fast convergence to steady state, underestimates the contribution to variability stemming from temporary contracts by approx. 30 \%. 

We believe our results are important in a broader context. Indeed, many European labor markets are also characterised by low flows and dualism, which makes the standard method unlikely to be a suitable tool for analysis. Nevertheless existing studies on Spain and France \citep{Silva2013, Hairault2015} do still rely on the steady state approximation. This it likely that these studies have underestimate the variance contribution from temporary contracts.

This paper continues as follows. In Section \ref{sec:litt} we review the existing literature on decomposition of labor market variation. In Section \ref{sec:method} we describe our new decomposing method and contracts is with the standard steady state method.In Section \ref{sec:data} we describe our data on the Swedish labor market and in \ref{sec:Results} we apply both our new method as well as the standard method on the dataset. Finally, Section \ref{sec:conclusion} concludes.
