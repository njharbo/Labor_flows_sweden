
to log-linearize the

an

Our paper complements the existing literature by proposing a method for decomposing labor market flows that simultaneously addresses time aggregation, non-steady state dynamics, and allows for an arbitrary number of labor market states. The method builds on the recognition that when the underlying data is generated by a continuous time Markov chain, the probability transition matrix is the matrix exponent of the matrix of bilateral labor market flows. With this formulation, it becomes easy to express a log-linearization around the labor market trend taking into account the whole history of labor market flows. The matrix formulation also allows us to judge the plausibility of the steady-state assumption by considering the second largest eigenvalue of the flow matrix which regulates the convergence rate to steady state.

We apply our methodology to the Swedish labor market data using monthly labor market survey data from 1987-2011, using a four-state model with permanent employment, temporary employment, unemployment, and inactivity. In our preferred specification, we find that xx% of variations in unemployment is caused by inflows to unemployment and yy% is caused by outflows.

The results show firstthat the distinction between permanent and temporary unemployment is important. There is a very large difference in the labor market dynamics between people in permanent and temporary employment. The flow rate to unemployment for permanently employed is zz% and for temporary employment is aa%. The distinction also matters for the decomposition exercise. We find that [??] all effects on unemployment from varying outflow is due to varying outflow to temporary employment. When it comes to losing jobs, bb% of the effect comes from people losing permanent employment and cc% come from people losing temporary employment.

Second, the result also shows that it is also important to consider non-steady state dynamics in our analysis. The eigenvalue analysis of the flow matrix suggests that the convergence rate to steady state in the Swedish labor market is only dd% per month. The importance of slow convergence is also revealed in the decomposition exercise. If we use the method of Silva & Grenno (2013) who use a steady state approximation of a dual labor market, we dramatically overestimate the impact of variations in inflows to unemployment from permanent employment. The reason is that the large stock of permanently employed means that small changes in inflow to unemployment rates have large effects on the steady-state unemployment rate. However, the small absolute value of flows means that the convergence rate to steady state for permanently employed is very slow, and the increase in inflows to unemployment is reversed before the low steady state is reached.
